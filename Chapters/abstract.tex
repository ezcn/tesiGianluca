% Chapter discussion

\chapter{Abstract}
    
 % Main chapter title

\label{Chapter0} % For referencing the chapter elsewhere, use \ref{Chapter4} 

%----------------------------------------------------------------------------------------

% Define some commands to keep the formatting separated from the content 
\newcommand{\keyword}[1]{\textbf{#1}}
\newcommand{\tabhead}[1]{\textbf{#1}}
\newcommand{\code}[1]{\texttt{#1}}
\newcommand{\file}[1]{\texttt{\bfseries#1}}
\newcommand{\option}[1]{\texttt{\itshape#1}}

%~~~~~~~~~~~~~~~~~~~~~~~~~~~~~~~~~~~~~~~~~~~~~~~~~~~~~~~~~~~~~~~~~~~~~~~~~~~~~~~~~~~~~~~




 Pregnancy Loss (PL), the spontaneous termination of a pregnancy before 24 weeks of gestation, occurs in 10-15\% of pregnancies. PL is often the result of chromosomal aneuploidies of the gametes but it can also have non-random genetic causes like small mutations (SNPs and indels), both de-novo or inherited from parents. Comparative genomic hybridization (CGH) detects variants of several thousand base pairs while targeted resequencing resolves point mutations. Both are currently the most accurate methods for the genetic analysis of PL, but are not sensitive to small variants (CGH), or do not target those located in non-coding regulatory regions (targeted resequencing). \\

\noindent
We aim to identify small-size genetic variants likely to cause PL using a predictive model that integrates whole-genome sequence data with functional annotations and gene networks relevant to embryonic development. In this pilot study we analyse whole-genome sequence of embryos from seventy women diagnosed with first (n=39, av.age 28.9 ) or recurrent (n=31, av.age 39.0) miscarriage. Prior to sequencing, samples were screened for aneuploidies using CGH and shallow sequencing. \\

\noindent
We understood the requirements to scale-up the project and obtained initial results from the analysis of these genomic sequences. We estimated that 30\% of collected samples are suitable for sequencing, with the rest presenting aneuploidies or quality issues and maternal contamination. Sequenced samples have on average 4M high-quality small variants of which 0.1\% are ranked as having a highly deleterious impact while 1.7\% have a moderate impact, and 2\% low. \\

\noindent
Within genic regions we identified twelve to twenty-seven genetic variants per sample of high (stop gain) and moderate (missense) impact with frequency lower than 5\% in gnomAD and 1000 Genomes populations, located in hundred-twelve genes. Of particular relevance we report findings in three genes. First, two samples share a stop gained mutation in the \textit{FMNL2} gene, that is expressed in the fetus in the cytoplasm of brain, spinal cord, and rectum
where it is involved in cytoskeleton organization. Secondly, all the six samples share a missense mutation that is rare in the general population (less than 0.01\%) in the \textit{GXYLT1} gene, a transmembrane protein involved in the O-glycosylation process. Finally, four samples have multiple hetrozygous missense mutations at seven sites in \textit{LAMA5}, a gene that regulates the attachment, migration, and organization of cells into tissues during embryonic development.\\

\noindent
These first results will be corroborated with a more comprehensive analysis that fully implement a predictive model and includes regulatory regions. I demonstrated that whole genome sequencing can help to clarify the causes of PL and provides essential indications for the realization of a larger study.

