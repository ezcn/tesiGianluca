\makeglossaries

\newglossaryentry{copy number variation}
{
    name=copy number variation,
    description={xxx }
}

\newglossaryentry{fastq}
{
    name=fastq,
    description={FASTQ format is a text-based format for storing both a biological sequence and its corresponding quality scores. Both the sequence letter and quality score are each encoded with a single ASCII character for brevity}
}

\newglossaryentry{reads}
{
    name=reads,
    description={In DNA sequencing, a read is an inferred sequence of base pairs (or base pair probabilities) corresponding to all or part of a single DNA fragment}
}

\newglossaryentry{coverage}
{
    name=coverage,
    description={The coverage is the number of unique reads that include a given nucleotide in the reconstructed sequence}
}

\newglossaryentry{bam}
{
    name=bam,
    description={Bam file is a text-based format originally for storing biological sequences aligned to a reference sequence}
}

\newglossaryentry{high performance computing}
{
    name=high performance computing,
    description={High-performance computing (HPC) is the use of super computers and parallel processing techniques for solving complex computational problems}
}

\newglossaryentry{genotype likelihood}
{
    name=genotype likelihood,
    description={XXX}
}

\newglossaryentry{vcf}
{
    name=vcf,
    description={The VCF file is a text file used in bioinformatics for storing gene sequence variations. The standard is currently in version 4.3, and it is made up of two parts: the header and the body. The header contains keywords that optionally semantically and syntactically describe the fields used in the body of the file, notably INFO, FILTER, and FORMAT. The body of VCF follows the header, and is tab separated into 8 mandatory columns and an unlimited number of optional columns that may be used to record other information about the sample(s)(\href{http://en.wikipedia.org/wiki/Variant_Call_Format}{wiki})}
}

\newglossaryentry{quality score}
{
    name=quality score,
    description={QUAL phred-scaled quality score for the assertion made in ALT. i.e. give -10log_10 prob(call in ALT is wrong)}
}

\newglossaryentry{code}
{
    name=code,
    description={XXX}
}

\newglossaryentry{dependencies}
{
    name=dependencies,
    description={XXX}
}
